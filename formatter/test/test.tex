\documentclass{report}
\usepackage{float}
\usepackage{graphicx}
\usepackage{amsthm}
\usepackage{amsmath, amssymb}
\usepackage{xparse}
\usepackage{xcolor}
\usepackage{dashrule}
\usepackage{enumitem}
\usepackage{changepage}
\usepackage{hyperref}

\setlength{\parindent}{0pt}

\NewDocumentEnvironment{theorem}{O{} O{}}
{\phantomsection\label{theorem:#1} \par\noindent\textbf{Theorem #1: #2}\par\vspace{0.5em} \setlength{\parindent}{0pt} \begin{adjustwidth}{1em}{0pt} \ignorespaces}
{\end{adjustwidth}\par\vspace{0.5em}}

\NewDocumentEnvironment{concept}{O{} O{}}
{\phantomsection\label{concept:#1} \par\noindent\textbf{Concept #1: #2}\par\vspace{0.5em} \setlength{\parindent}{0pt} \begin{adjustwidth}{1em}{0pt} \ignorespaces}
{\end{adjustwidth}\par\vspace{0.5em}}

\NewDocumentEnvironment{definition}{O{} O{}}
{\phantomsection\label{definition:#1} \par\noindent\textbf{Definition #1: #2}\par\vspace{0.5em} \setlength{\parindent}{0pt} \begin{adjustwidth}{1em}{0pt} \ignorespaces}
{\end{adjustwidth}\par\vspace{0.5em}}

\NewDocumentEnvironment{example}{O{} O{}}
{\phantomsection\label{example:#1} \par\noindent\textbf{Example #1: #2}\par\vspace{0.5em} \setlength{\parindent}{0pt} \begin{adjustwidth}{1em}{0pt} \ignorespaces}
{\end{adjustwidth}\par\vspace{0.5em}}

\NewDocumentEnvironment{proofsketch}{O{} O{}}
{\par\noindent\textbf{Proof #1: #2}\par\vspace{0.5em} \setlength{\parindent}{0pt} \begin{adjustwidth}{1em}{0pt} \ignorespaces}
{\end{adjustwidth}\par\vspace{0.5em}}

\begin{document}

\title{Custom Theorems and Lists in LaTeX}
\author{}
\date{}
\maketitle

\part{SDF}

% === Chapter 1 ===
\chapter{Introduction}


Go to \hyperref[concept:13.8]{this hidden section}.

\begin{theorem}[1.3][Pythagoras Theorem]
    In a right-angled triangle, the square of the hypotenuse is equal to the sum of the squares of the other two sides.

    $$
    a^2 = b^2 + c^2
    $$
\end{theorem}

\begin{concept}[2.1][Probability]
    The probability of an event is a measure of the likelihood that the event will occur.
\end{concept}

% === Chapter 2 ===
\chapter{Images and Lists}


\section{Lists in LaTeX}

\subsection{Unordered List}
\begin{itemize}
    \item First item
    \item Second item
    \item Third item
\end{itemize}

\subsection{Ordered List}
\begin{enumerate}
    \item First item
    \item Second item
    \item Third item
\end{enumerate}

% === Chapter 3 ===
\chapter{Code Blocks}

\section{Simple Code (Verbatim)}
\begin{verbatim}
printf("Hello World");
\end{verbatim}

\chapter{Optimization Problem}

\section{Definition of Optimization Problem}

\begin{definition}[1.1][Optimization Problem]
    In an **optimization problem**, we minimize or maximize a function value, possibly subject to constraints.

    $$
    \begin{array}{ll}
    \underset{\theta \in \mathbb{R}^{p}}{\operatorname{minimize}} & f(\theta) \\[1em]
    \text {subject to} & h_{1}(\theta) = 0, \quad h_{2}(\theta) = 0, \quad \dots, \quad h_{m}(\theta) = 0, \\[0.5em]
    & g_{1}(\theta) \leq 0, \quad g_{2}(\theta) \leq 0, \quad \dots, \quad g_{n}(\theta) \leq 0.
    \end{array}
    $$

    \begin{itemize}
        \item Decision variable: $\theta$
        \item Objective function: $f$
        \item Equality constraint: $h_{i}(\theta)=0$ for $i=1, \ldots, m$
        \item Inequality constraint: $g_{j}(\theta) \leq 0$ for $j=1, \ldots, n$
    \end{itemize}
\end{definition}

In machine learning (ML), we often minimize a "loss", but sometimes we maximize the "likelihood".
In any case, minimization and maximization are equivalent since

$$
\text { maximize } f(\theta) \quad \Leftrightarrow \quad \text { minimize }-f(\theta).
$$

\begin{definition}[1.2][Feasible Point and Constraints]
    $\theta \in \mathbb{R}^{p}$ is a **feasible point** if it satisfies all constraints:

    $$
    \begin{array}{cc}
    h_{1}(\theta)=0 & g_{1}(\theta) \leq 0 \\
    \vdots & \vdots \\
    h_{m}(\theta)=0 & g_{n}(\theta) \leq 0
    \end{array}
    $$

    Optimization problem is **infeasible** if there is no feasible point.

    An optimization problem with no constraint is called an **unconstrained optimization problem**. Optimization problems with constraints is called a **constrained optimization problem**.
\end{definition}

\begin{definition}[1.3][Optimal Value and Solution]
    **Optimal value** of an optimization problem is

    $$
    p^{\star}=\inf \left\{f(\theta) \mid \theta \in \mathbb{R}^{n}, \theta \text { feasible }\right\}
    $$

    \begin{itemize}
        \item $p^{\star}=\infty$ if problem is infeasible
        \item $p^{\star}=-\infty$ is possible
        \item In ML, it is often a priori clear that $0 \leq p^{\star}<\infty$.
    \end{itemize}

    If $f\left(\theta^{\star}\right)=p^{\star}$, we say $\theta^{\star}$ is a **solution** or $\theta^{\star}$ is **optimal**.  
    A solution may or may not exist, and a solution may or may not be unique.
\end{definition}

\section{Examples of Optimization Problem}

\begin{example}[1.4][Curve Fitting]
    Consider setup with data $X_{1}, \ldots, X_{N}$ and corresponding labels $Y_{1}, \ldots, Y_{N}$ satisfying the relationship

    $$
    Y_{i}=f_{\star}\left(X_{i}\right)+\text { error }
    $$

    for $i=1, \ldots, N$. Hopefully, "error" is small. True function $f_{\star}$ is unknown.

    Goal is to find a function (curve) $f$ such that $f \approx f_{\star}$.
\end{example}

\begin{example}[1.5][Least-Squares Minimization]
    \begin{itemize}
        \item **Problem**
    \end{itemize}

    $$
    \underset{\theta \in \mathbb{R}^{p}}{\operatorname{minimize}} \quad \frac{1}{2}\|X \theta-Y\|^{2}
    $$

    where $X \in \mathbb{R}^{N \times p}$ and $Y \in \mathbb{R}^{N}$. Equivalent to

    $$
    \underset{\theta \in \mathbb{R}^{p}}{\operatorname{minimize}} \frac{1}{2} \sum_{i=1}^{N}\left(X_{i}^{\top} \theta-Y_{i}\right)^{2}
    $$

    where $X=\left[\begin{array}{c}X_{1}^{\top} \\ \vdots \\ X_{N}^{\top}\end{array}\right]$ and $Y=\left[\begin{array}{c}Y_{1} \\ \vdots \\ Y_{N}\end{array}\right]$.

    \hrule

    \begin{itemize}
        \item **Solution**
    \end{itemize}

    To solve

    $$
    \underset{\theta \in \mathbb{R}^{p}}{\operatorname{minimize}} \frac{1}{2}\|X \theta-Y\|^{2}
    $$

    take gradient and set it to $0$.

    $$
    \nabla_{\theta} \frac{1}{2}\|X \theta-Y\|^{2}=X^{\top}(X \theta-Y)
    $$

    $$
    \begin{gathered}
    X^{\top}\left(X \theta^{\star}-Y\right)=0 \\
    \theta^{\star}=\left(X^{\top} X\right)^{-1} X^{\top} Y
    \end{gathered}
    $$

    Here, we assume $X^{\top} X$ is invertible.
\end{example}

\begin{concept}[1.6][Least squares is an instance of curve fitting.]
    Define $f_{\theta}(x)=x^{\top} \theta$.
    Then LS becomes

    $$
    \underset{\theta \in \mathbb{R}^{p}}{\operatorname{minimize}} \frac{1}{2} \sum_{i=1}^{N}\left(f_{\theta}\left(X_{i}\right)-Y_{i}\right)^{2}
    $$

    and the solution hopefully satisfies

    $$
    Y_{i}=f_{\theta}\left(X_{i}\right)+\text { small. }
    $$

    Since $X_{i}$ and $Y_{i}$ is assumed to satisfy

    $$
    Y_{i}=f_{\star}\left(X_{i}\right)+\text { error }
    $$

    we are searching over linear functions (linear curves) $f_{\theta}$ that best fit (approximate) $f_{\star}$.
\end{concept}

\section{Local and Global Minimum}

\begin{definition}[1.7][Local vs Global Minima]
    $\theta^{\star}$ is a **local minimum** if $f(\theta) \geq f\left(\theta^{\star}\right)$ for all feasible $\theta$ within a small neighborhood.

    $\theta^{\star}$ is a **global minimum** if $f(\theta) \geq f\left(\theta^{\star}\right)$ for all feasible $\theta$.

    \begin{figure}[H]
        \centering
        \includegraphics[width=0.5\textwidth]{./1.1.jpg}
    \end{figure}

    In the worst case, finding the global minimum of an optimization problem is difficult.
    However, in deep learning, optimization problems are often "solved" without any guarantee of global optimality.
\end{definition}

\chapter{Basics of Monte Carlo}

\section{Monte Carlo Estimation}

\begin{definition}[13.1][Monte Carlo Estimation]
    Consider IID data $X_{1}, \ldots, X_{N} \sim f$.
    Let $\phi(X) \geq 0$ be some function. (The assumption $\phi(X) \geq 0$ can be relaxed.)
    Consider the problem of estimating

    $$
    I=\mathbb{E}_{X \sim f}[\phi(X)]=\int \phi(x) f(x) d x
    $$

    One commonly uses

    $$
    \hat{I}_{N}=\frac{1}{N} \sum_{i=1}^{N} \phi\left(X_{i}\right)
    $$

    to estimate $I$, which is called \textbf{monte carlo estimation}.
    After all, $\mathbb{E}\left[\hat{I}_{N}\right]=I$ and $\hat{I}_{N} \rightarrow I$ by the law of large numbers.
    (Convergence in probability by weak law of large numbers and almost sure convergence by strong law of large numbers.)
\end{definition}

\begin{concept}[13.2][Evidence of Convergence for Monte Carlo Estimation]
    We can quantify convergence with variance:

    $$
    \operatorname{Var}_{X \sim f}\left(\hat{I}_{N}\right)=\sum_{i=1}^{N} \operatorname{Var}_{X_{i} \sim f}\left(\frac{\phi\left(X_{i}\right)}{N}\right)=\frac{1}{N} \operatorname{Var}_{X \sim f}(\phi(X))
    $$

    In other words

    $$
    \mathbb{E}\left[\left(\hat{I}_{N}-I\right)^{2}\right]=\frac{1}{N} \operatorname{Var}_{X \sim f}(\phi(X))
    $$

    and

    $$
    \mathbb{E}\left[\left(\hat{I}_{N}-I\right)^{2}\right] \rightarrow 0
    $$

    as $N \rightarrow \infty$.
    So, $\hat{I}_{N} \rightarrow I$ in $L^2$ provided that $\operatorname{Var}_{X \sim f}(\phi(X)) < \infty$.
\end{concept}

\begin{definition}[13.3][Empirical Risk Minimization (ERM)]
    In machine learning and statistics, we often wish to solve

    $$
    \underset{\theta \in \Theta}{\operatorname{minimize}} \quad \mathcal{L}(\theta)
    $$

    where the objective function

    $$
    \mathcal{L}(\theta)=\mathbb{E}_{X \sim p_{X}}\left[\ell\left(f_{\theta}(X), f_{\star}(X)\right)\right]
    $$

    Is the (true) risk. However, the evaluation of $\mathbb{E}_{X \sim p_{X}}$ is impossible (if $p_{X}$ is unknown) or intractable (if $p_{X}$ is known but the expectation has no closed-form solution). Therefore, we define the proxy loss function

    $$
    \mathcal{L}_{N}(\theta)=\frac{1}{N} \sum_{i=1}^{N} \ell\left(f_{\theta}\left(X_{i}\right), f_{\star}\left(X_{i}\right)\right)
    $$

    which we call the empirical risk, and solve

    $$
    \underset{\theta \in \Theta}{\operatorname{minimize}} \quad \mathcal{L}_{N}(\theta)
    $$

    This is called \textbf{empirical risk minimization (ERM)}. The idea is that

    $$
    \mathcal{L}_{N}(\theta) \approx \mathcal{L}(\theta)
    $$

    with high probability, so minimizing $\mathcal{L}_{N}(\theta)$ should be similar to minimizing $\mathcal{L}(\theta)$.
\end{definition}

\begin{concept}[13.4][Evidence of Convergence for Empirical Risk Minimization]
    Technical note) The law of large numbers tells us that

    $$
    \mathbb{P}\left(\left|\mathcal{L}_{N}(\theta)-\mathcal{L}(\theta)\right|>\varepsilon\right)=\text { small }
    $$

    for any given $\theta$, but we need

    $$
    \mathbb{P}\left(\sup _{\theta \in \Theta}\left|\mathcal{L}_{N}(\theta)-\mathcal{L}(\theta)\right|>\varepsilon\right)=\text { small }
    $$

    for all compact $\Theta$ in order to conclude that the argmins of the two losses to be similar. These types of results are established by a uniform law of large numbers.
\end{concept}

\section{Importance Sampling}

\begin{definition}[13.5][Importance Sampling (IS)]
    \textbf{Importance sampling (IS)} is a technique for reducing the variance of a Monte Carlo estimator.

    Key insight of important sampling:

    $$
    I=\mathbb{E}_{X \sim f}[\phi(X)]=\int \phi(x) f(x) d x=\int \frac{\phi(x) f(x)}{g(x)} g(x) d x=\mathbb{E}_{X \sim g}\left[\frac{\phi(X) f(X)}{g(X)}\right]
    $$

    (We do have to be mindful of division by 0.) Then

    $$
    \hat{I}_{N}=\frac{1}{N} \sum_{i=1}^{N} \phi\left(X_{i}\right) \frac{f\left(X_{i}\right)}{g\left(X_{i}\right)}
    $$

    with $X_{1}, \ldots, X_{N} \sim g$ is also an estimator of $I$.
    Indeed, $\mathbb{E}\left[\hat{I}_{N}\right]=I$ and $\hat{I}_{N} \rightarrow I$.
    The weight $\frac{f(x)}{g(x)}$ is called the \textbf{likelihood ratio} or the \textbf{Radon-Nikodym derivative}.

    So we can use samples from $g$ to compute expectation with respect to $f$.
\end{definition}

\begin{example}[13.6][IS Example]
    Consider the setup of estimating the probability

    $$
    \mathbb{P}(X>3)=0.00135
    $$

    where $X \sim \mathcal{N}(0,1)$. If we use the regular Monte Carlo estimator

    $$
    \hat{I}_{N}=\frac{1}{N} \sum_{i=1}^{N} \mathbf{1}_{\left\{X_{i}>3\right\}}
    $$

    where $X_{i} \sim \mathcal{N}(0,1)$, if $N$ is not sufficiently large, we can have $\hat{I}_{N}=0$. Inaccurate estimate.

    If we use the IS estimator

    $$
    \hat{I}_{N}=\frac{1}{N} \sum_{i=1}^{N} \mathbf{1}_{\left\{Y_{i}>3\right\}} \exp \left(\frac{\left(Y_{i}-3\right)^{2}-Y_{i}^{2}}{2}\right)
    $$

    where $Y_{i} \sim \mathcal{N}(3,1)$, having $\hat{I}_{N}=0$ is much less likely. Estimate is much more accurate.
\end{example}

\begin{concept}[13.7][Optimal Sampling Distribution]
    Benefit of IS quantified by with variance:

    $$
    \operatorname{Var}_{X \sim g}\left(\hat{I}_{N}\right)=\sum_{i=1}^{N} \operatorname{Var}_{X \sim g}\left(\frac{\phi\left(X_{i}\right) f\left(X_{i}\right)}{n g\left(X_{i}\right)}\right)=\frac{1}{N} \operatorname{Var}_{X \sim g}\left(\frac{\phi(X) f(X)}{g(X)}\right)
    $$

    If $\operatorname{Var}_{X \sim g}\left(\frac{\phi(X) f(X)}{g(X)}\right)<\operatorname{Var}_{X \sim f}(\phi(X))$, then IS provides variance reduction.

    We call $g$ the importance or sampling distribution. Choosing $g$ poorly can increase the variance. What is the best choice of $g$ ?

    \par\noindent\textcolor{gray}{\hdashrule{\textwidth}{0.4pt}{1pt 2pt}}

    The sampling distribution

    $$
    g(x)=\frac{\phi(x) f(x)}{I}
    $$

    makes $\operatorname{Var}_{X \sim g}\left(\frac{\phi(X) f(X)}{g(X)}\right)=\operatorname{Var}_{X \sim g}(I)=0$ and therefore is optimal. ($I$ serves as the normalizing factor that ensures the density $g$ integrates to 1.)

    Problem: Since we do not know the normalizing factor $I$, the answer we wish to estimate, sampling from $g$ is usually difficult.
\end{concept}

\begin{concept}[13.8][Optimized / Trained Sampling Distribution]
    Instead, we consider the optimization problem

    $$
    \underset{g \in \mathcal{G}}{\operatorname{minimize}} \quad D_{\mathrm{KL}}\left(g \| \frac{\phi f}{I}\right)
    $$

    and compute a suboptimal, but good, sampling distribution within a class of sampling distributions $\mathcal{G}$. (In ML, $\mathcal{G}=\left\{g_{\theta} \mid \theta \in \Theta\right\}$ is parameterized by neural networks.)

    Importantly, this optimization problem does not require knowledge of $I$.

    $$
    \begin{aligned}
    D_{\mathrm{KL}}\left(g_{\theta} \| \phi f / I\right) & =\mathbb{E}_{X \sim g_{\theta}}\left[\log \left(\frac{I g_{\theta}(X)}{\phi(X) f(X)}\right)\right] \\
    & =\mathbb{E}_{X \sim g_{\theta}}\left[\log \left(\frac{g_{\theta}(X)}{\phi(X) f(X)}\right)\right]+\log I \\
    & =\mathbb{E}_{X \sim g_{\theta}}\left[\log \left(\frac{g_{\theta}(X)}{\phi(X) f(X)}\right)\right]+\text { constant independent of } \theta
    \end{aligned}
    $$

    How do we compute stochastic gradients?
\end{concept}

\section{Log-Derivative Trick}

\begin{definition}[13.9][Log-Derivative Trick]
    Generally, consider the setup where we wish to solve

    $$
    \underset{\theta \in \mathbb{R}^{p}}{\operatorname{minimize}} \mathbb{E}_{X \sim f_{\theta}}[\phi(X)]
    $$

    with SGD.
    (Previous situation (\hyperref[concept:13.8]{Concept 13.8}) had $\theta$-dependence both on and inside the expectation. For now, let's simplify the problem so that $\phi$ does not depend on $\theta$.)

    Incorrect gradient computation:

    $$
    \nabla_{\theta} \mathbb{E}_{X \sim f_{\theta}}[\phi(X)] \stackrel{?}{=} \mathbb{E}_{X \sim f_{\theta}}\left[\nabla_{\theta} \phi(X)\right]=\mathbb{E}_{X \sim f_{\theta}}[0]=0
    $$

    Correct gradient computation:

    $$
    \begin{aligned}
    \nabla_{\theta} \mathbb{E}_{X \sim f_{\theta}}[\phi(X)] & =\nabla_{\theta} \int \phi(x) f_{\theta}(x) d x=\int \phi(x) \nabla_{\theta} f_{\theta}(x) d x \\
    & =\int \phi(x) \frac{\nabla_{\theta} f_{\theta}(x)}{f_{\theta}(x)} f_{\theta}(x) d x=\mathbb{E}_{X \sim f_{\theta}}\left[\phi(X) \frac{\nabla_{\theta} f_{\theta}(X)}{f_{\theta}(X)}\right] \\
    & =\mathbb{E}_{X \sim f_{\theta}}\left[\phi(X) \nabla_{\theta} \log \left(f_{\theta}(X)\right)\right]
    \end{aligned}
    $$

    Therefore, $\phi(X) \nabla_{\theta} \log \left(f_{\theta}(X)\right)$ with $X \sim f_{\theta}$ is a stochastic gradient of the loss function. This technique is called the log-derivative trick, the likelihood ratio gradient$^{\#}$, or REINFORCE$^{\star}$.

    Formula with the log-derivative $\left(\nabla_{\theta} \log (\cdot)\right)$ is convenient when dealing with Gaussians, or more generally exponential families, since the densities are of the form

    $$
    f_{\theta}(x)=h(x) \exp (\text {function of } \theta)
    $$

    (${}^{\#}$P. W. Glynn, Likelihood ratio gradient estimation for stochastic systems, Communications of the ACM, 1990.\\
    ${}^{\star}$R. J. Williams, Simple statistical gradient-following algorithms for connectionist reinforcement learning. Machine Learning, 1992.)
\end{definition}

\begin{example}[13.10][Log-Derivative Trick Example]
    Learn $\mu \in \mathbb{R}^{2}$ to minimize the objective below.

    $$
    \underset{\mu \in \mathbb{R}^{2}}{\operatorname{minimize}} \mathbb{E}_{X \sim \mathcal{N}(\mu, I)}\left\|X-\binom{5}{5}\right\|^{2}
    $$

    Then the loss function is

    $$
    \mathcal{L}(\mu)=\mathbb{E}_{X \sim \mathcal{N}(\mu, I)}\left\|X-\binom{5}{5}\right\|^{2}=\int\left\|x-\binom{5}{5}\right\|^{2} \frac{1}{2 \pi} \exp \left(-\frac{1}{2}\|x-\mu\|^{2}\right) d x
    $$

    And, using $X_{1}, \ldots, X_{B} \sim \mathcal{N}(\mu, I)$, we have stochastic gradients

    $$
    \nabla_{\mu} \mathcal{L}(\mu)=\mathbb{E}_{X \sim \mathcal{N}(\mu, I)}\left[\left\|x-\binom{5}{5}\right\|^{2} \nabla_{\mu}\left(-\frac{1}{2}\|x-\mu\|^{2}\right)\right] \approx \frac{1}{B} \sum_{i=1}^{B}\left\|X_{i}-\binom{5}{5}\right\|^{2}\left(X_{i}-\mu\right)
    $$

    These stochastic gradients have large variance and thus SGD is slow.

    \begin{figure}[H]
        \centering
        \includegraphics[width=0.5\textwidth]{13.1.png}
    \end{figure}
\end{example}

\section{Reparameterization Trick}

\begin{definition}[13.11][Reparameterization Trick]
    The reparameterization trick (RT) or the pathwise derivative (PD) relies on the key insight.

    $$
    \mathbb{E}_{X \sim \mathcal{N}\left(\mu, \sigma^{2}\right)}[\phi(X)]=\mathbb{E}_{Y \sim \mathcal{N}(0,1)}[\phi(\mu+\sigma Y)]
    $$

    Gradient computation:

    $$
    \begin{aligned}
    \nabla_{\mu, \sigma} \mathbb{E}_{X \sim \mathcal{N}\left(\mu, \sigma^{2}\right)}[\phi(X)] & =\mathbb{E}_{Y \sim \mathcal{N}(0,1)}\left[\nabla_{\mu, \sigma} \phi(\mu+\sigma Y)\right]=\mathbb{E}_{Y \sim \mathcal{N}(0,1)}\left[\phi^{\prime}(\mu+\sigma Y)\left[\begin{array}{c}
    1 \\
    Y
    \end{array}\right]\right] \\
    & \approx \frac{1}{B} \sum_{i=1}^{B} \phi^{\prime}\left(\mu+\sigma Y_{i}\right)\left[\begin{array}{c}
    1 \\
    Y_{i}
    \end{array}\right], \quad Y_{1}, \ldots, Y_{B} \sim \mathcal{N}(0, I)
    \end{aligned}
    $$

    RT is less general than log-derivative trick, but it usually produces stochastic gradients with lower variance.
\end{definition}

\begin{example}[13.12][Reparameterization Trick Example]
    Consider the same example as before

    $$
    \mathcal{L}(\mu)=\mathbb{E}_{X \sim \mathcal{N}(\mu, I)}\left\|X-\binom{5}{5}\right\|^{2}=\mathbb{E}_{Y \sim \mathcal{N}(0, I)}\left\|Y+\mu-\binom{5}{5}\right\|^{2}
    $$

    Gradient computation:

    $$
    \begin{aligned}
    \nabla_{\mu} \mathcal{L}(\mu) & =\mathbb{E}_{Y \sim \mathcal{N}(0, I)} \nabla_{\mu}\left\|Y+\mu-\binom{5}{5}\right\|^{2}=2 \mathbb{E}_{Y \sim \mathcal{N}(0, I)}\left(Y+\mu-\binom{5}{5}\right) \\
    & \approx \frac{2}{B} \sum_{i=1}^{B}\left(Y_{i}+\mu-\binom{5}{5}\right), \quad Y_{1}, \ldots, Y_{B} \sim \mathcal{N}(0, I)
    \end{aligned}
    $$

    These stochastic gradients have smaller variance and thus SGD is faster.
\end{example}

\begin{example}[13.13][Log Derivative Trick vs Reparameterization Trick]
    The image below is the result of SGD with the computed gradients by \hyperref[example:13.10]{Example 13.10} and \hyperref[example:13.12]{Example 13.12}.

    \begin{figure}[H]
        \centering
        \includegraphics[width=1.0\textwidth]{13.2.png}
    \end{figure}
\end{example}






\end{document}
