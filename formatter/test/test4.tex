\documentclass{beamer}
\makeatletter
\@ifclassloaded{beamer}{\usecolortheme{orchid}}{\usepackage{beamerarticle}}
\makeatother

\NewDocumentEnvironment{preuve}{O{Preuve}}{\begin{block}{#1}}{\end{block}}

\begin{document}
\begin{frame}
\begin{alertblock}{Berge, 1957}
Un couplage $C$ est de cardinal maximal si et seulement s'il n'admet pas de chemin augmentant.
\end{alertblock}
\pause
\begin{preuve}[Preuve du sens non couvert par le lemme]
some already long text
\mergeFrame{preuve}{block}
\end{preuve}
\end{frame}

\begin{frame}
\begin{block}{}

Soit $p$ un chemin maximal dans $K$ : \uncover<2->{sa première et sa dernière arêtes sont dans $C_2$.}
\end{block}

Remarque : ce lemme et ce théorème n'utilisent pas le fait que le graphe est biparti.\\
Cette propriété va cependant permettre de faciliter la recherche d'un chemin augmentant.
\end{frame}
\end{document}